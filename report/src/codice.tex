\section{Codice}

\subsection{Struttura del progetto}

Il progetto CityBus è suddiviso in due macro-componenti principali:

\begin{itemize}
    \item Backend: Implementa la logica di business, l’accesso ai dati, la gestione delle linee, fermate, corse e il calcolo dei percorsi.
    \item Frontend: Applicazione web (SPA, Vue.js) per la consultazione, gestione e simulazione del servizio.
\end{itemize}

\subsection{Backend}

\subsubsection{Gestione delle corse e simulazione}

La gestione delle corse (\texttt{BusRide}) rappresenta uno degli aspetti centrali. Questo modulo consente di creare, aggiornare, monitorare e simulare le corse degli autobus in tempo reale, offrendo sia funzionalità amministrative che strumenti per la visualizzazione dinamica lato utente.

La gestione delle corse è affidata a due componenti principali: \texttt{BusRideManager} e \texttt{RideAgent}. Questi orchestrano la simulazione delle corse, il loro avanzamento e la comunicazione in tempo reale con i client tramite WebSocket e Redis.
Il \texttt{BusRideManager} mantiene una lista di agenti (\texttt{RideAgent}) attivi e, tramite un ciclo periodico, verifica quali corse devono essere avviate o terminate. Per ogni corsa attiva, viene creato un nuovo agente che ne gestisce la simulazione.

\begin{lstlisting}[language=JavaScript]
// Estratto da BusRideManager.js
class BusRideManager {
    constructor() { this.agents = []; }
    async init(io) {
        this.io = io;
        this.startLoop();
    }
    async startLoop() {
        setInterval(async () => {
            // Per ogni corsa programmata
            const lines = await BusLine.find().populate('directions');
            lines.map(line =>
                line.directions.map(direction =>
                    direction.timetable.map(async time => {
                        const ride = await BusRide.findOne({ /* ... */ });
                        const agent = this.agents.find(ag => ag.ride._id.toString() == ride._id.toString());
                        if (!agent && ride) {
                            const newAgent = new RideAgent(ride, this.io);
                            newAgent.start();
                            this.agents.push(newAgent);
                        }
                    })
                )
            );
        }, 10000);
    }
}
\end{lstlisting}

\subsubsection{RideAgent e simulazione}

Ogni \texttt{RideAgent} simula l’avanzamento della corsa, calcola la posizione del bus e comunica gli aggiornamenti tramite WebSocket. Se la corsa termina, l’agente la marca come conclusa e si ferma.

\begin{lstlisting}[language=JavaScript]
// Estratto da RideAgent.js
class RideAgent {
    constructor(ride, io) { /* ... */ }
    async start() {
        this.interval = setInterval(async () => {
            const position = await getBusPosition(this.ride);
            if (!position) {
                this.ride.stops.forEach(stop => stop.isBusPassed = true);
                await this.ride.save();
                this.stop();
                return;
            }
            this.socket.emit("put", JSON.stringify(position));
        }, 1000);
    }
    stop() {
        clearInterval(this.interval);
        if (this.socket) this.socket.disconnect();
    }
}
\end{lstlisting}

\subsubsection{Aggiornamento tramite WebSocket e Redis}

Gli aggiornamenti di posizione vengono inviati tramite WebSocket ai client. Il controller (\texttt{socketsController.js}) riceve i dati, aggiorna lo stato della corsa e li propaga tramite Redis per garantire scalabilità e sincronizzazione tra più istanze.

\begin{lstlisting}
// Estratto da socketsController.js
socket.on('put', async (position) => {
    const rideData = await calculateRealTimeRideData({rideId, position: JSON.parse(position)});
    await rideDataProvider.setRide(rideId, rideData); // Aggiorna Redis
});
rideDataEvent.onMessage((rideData) => {
    socket.emit('ride_update', rideData); // Inoltra ai client
});
\end{lstlisting}

\subsubsection{Navigazione}

La funzionalità di navigazione è affidata al modulo \texttt{pathFinder}, che implementa la logica per il calcolo dei percorsi ottimali tra due punti della rete di trasporto pubblico. Il cuore di questo componente è la funzione \texttt{getNavigationPath}, che sfrutta l’algoritmo Connection Scan (CSA) per trovare il percorso più efficiente combinando tratte a piedi e tratte in autobus.

Per ogni richiesta di percorso, il sistema:
\begin{itemize}
    \item Identifica le fermate più vicine ai punti di partenza e arrivo tramite query geospaziali.
    \item Costruisce una lista di “connessioni” che rappresentano sia i tratti a piedi (dall’utente alla fermata e viceversa) sia le tratte in autobus, includendo orari di partenza e arrivo.
    \item Per ogni connessione bus, calcola gli orari effettivi in base alla tabella oraria della linea e della direzione.
\end{itemize}

\begin{lstlisting}
// Estratto da pathFinder.js
conns.push({
    '@id': connectionId,
    travelMode: 'bus',
    departureStop: departureStopId,
    departureTime: departureTime,
    arrivalStop: arrivalStopId,
    arrivalTime: arrivalTime
});
// ...aggiunta delle connessioni bus e a piedi...
\end{lstlisting}

Le connessioni vengono ordinate per orario di partenza e passate a un planner CSA, che elabora il percorso ottimale tramite uno stream. Il risultato è una sequenza di tratte (bus e piedi) che minimizza il tempo totale di viaggio.

\begin{lstlisting}
// Esecuzione del planner CSA
const planner = new csa.BasicCSA({
    departureStop: departureStop,
    arrivalStop: arrivalStop,
    departureTime: departureTime
});
const connectionsReadStream = Readable.from(conns);
connectionsReadStream.pipe(planner);

planner.on("result", function (result) {
    resolve(result);
});
\end{lstlisting}

Il percorso calcolato viene restituito come lista di tratte, ciascuna con informazioni su tipo di spostamento, orari, e fermate. Questa logica è utilizzata dal controller delle rotte per fornire agli utenti itinerari dettagliati e ottimizzati.

\subsection{Frontend}

\subsubsection{Autenticazione e Registrazione: gestione dei Bearer Token}

L'autenticazione e la registrazione nel frontend sono progettate per garantire sicurezza e praticità, facendo uso di Bearer Token (JWT) per la gestione delle sessioni utente. L'intero flusso si basa su chiamate HTTP sicure e sulla conservazione dei token in locale, con rinnovo automatico quando necessario.

Il login è gestito tramite \texttt{AuthenticationService}. Al successo, il backend restituisce un JWT e un token di rinnovo (\texttt{renewToken}). Questi vengono salvati nello store locale e utilizzati per autenticare tutte le richieste successive tramite l'header \texttt{Authorization: Bearer <jwt>}.

\begin{lstlisting}
// src/service/AuthenticationService.js
export const AuthenticationService = {
    async login(email, password) {
        const res = await requests.post(`/auth/session`, { data: {email, password} })
        if (res.status == 201) {
            return {jwt: res.data.jwt, renewToken: res.data.renewToken}
        }
        throw 'Email o password non validi'
    }
}
\end{lstlisting}

\begin{lstlisting}
// src/stores/authentication.js
export const useAuthenticationStore = defineStore('authentication', {
    state: () => ({
        jwt: localStorage.getItem('jwt'),
        renewToken: localStorage.getItem('renewToken')
    }),
    actions: {
        setTokens(jwt, renewToken) {
            this.jwt = jwt
            localStorage.setItem('jwt', jwt)
            this.renewToken = renewToken
            localStorage.setItem('renewToken', renewToken)
        },
        deleteTokens() {
            localStorage.removeItem('jwt')
            localStorage.removeItem('renewToken')
        }
    }
});
\end{lstlisting}

Prima di ogni richiesta autenticata, il frontend verifica la validità del JWT. Se il token è scaduto, viene automaticamente rinnovato utilizzando il \texttt{renewToken}. Tutte le richieste protette includono l'header \texttt{Authorization} con il Bearer Token.

\begin{lstlisting}
// src/lib/requests.js
async function getAuthorizationHeader() {
    const authStore = useAuthenticationStore()
    const decoded = jwtDecode(authStore.jwt)
    if (decoded.exp < Math.floor(Date.now() / 1000) ) {
        await renewAuthenticationToken(authStore);
    }
    return `Bearer ${authStore.jwt}`
}

async function request(method, endpoint, data, authenticated) {
    const headers = {}
    if (authenticated) {
        headers['Authorization'] = await getAuthorizationHeader()
    }
    return await instance.request({
        url: endpoint,
        method: method,
        data: data,
        headers: headers
    })
}
\end{lstlisting}